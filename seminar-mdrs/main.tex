\documentclass{article}
\usepackage[left=3cm,right=3cm,top=2cm,bottom=2cm]{geometry}
\usepackage{amsmath}
\usepackage{amssymb}
\usepackage{amsthm}
\usepackage{hyperref}
\usepackage{colonequals}
\usepackage{xfrac}

\usepackage{graphicx}
\usepackage{subfig}

\setlength{\parindent}{0mm}

\newcommand{\R}[0]{\mathbb{R}}
\newcommand{\N}[0]{\mathbb{N}}
\newcommand{\T}[0]{\mathcal{T}}
\newcommand{\NB}[0]{\mathcal{N}}
\newcommand*{\logeq}{\ratio\Leftrightarrow}
\newcommand*{\longeq}{\ratio\Longleftrightarrow}
\newcommand*{\bigdot}{\mathpalette\bigcdot@{.5}}

\DeclareRobustCommand{\loongrightarrow}{%
  \DOTSB\relbar\joinrel\relbar\joinrel\rightarrow
}

\title{Seminar thesis - Fundamental groups as limits of discrete fundametal groups}
\author{Yannik Höll}
\date{\today}

\newtheoremstyle{break}%
{7pt}{7pt}%
{}{}%
{\bfseries}{:}% % Note that final punctuation is omitted.
{\newline}{}

\theoremstyle{break}
\newtheorem{thm}{Theorem}[section]

\theoremstyle{break}
\newtheorem{defin}[thm]{Definition}
\newtheorem{rem}[thm]{Remark}
\newtheorem{lemma}[thm]{Lemma}

\begin{document}

\nocite{*}

\maketitle

% TODO: add introduction with a summary of the main questions of the paper 
% i.e. can the fundamental group be recovered as a limit of discrete fundametal groups 

\section{Basic topological definitions} \label{section-basic-defs}

% TODO: only some defins, ref to def of topology, topological space, etc. 

\textbf{Notation:}
\begin{enumerate}
  \item Let $Y$ be a topological space with the topology $\T_Y$,
  \item Let $X$ be a metric space with the metric $d_X$ which is also a topological space with the topology induced by its metric,
  \item If $Z_1, Z_2$ are topological spaces then $C(Z_1, Z_2)$ is the set of all continous maps from $Z_1$ to $Z_2$,
  \item Let $y \in Y$ then $\NB_y$ is the set of all neighbourhoods of the element $y$.
  \item Let $\varepsilon \in \R_{>0}$ and let $x \in X$: $B_{\varepsilon}(x) := \{ y\in X \: | \: d_X(x, y) < \varepsilon\}$
\end{enumerate}

\begin{defin}
  A \textbf{path} $\gamma$ in $Y$ is a continuous map $\R \supseteq [0,1] \to Y$.
\end{defin}

\begin{defin}
  Let $\gamma_1, \gamma_2 \in C([0, 1], Y)$. A \textbf{free homotopy} $H: [0,1]^2 \to Y$ between $\gamma_1$ and $\gamma_2$ is a continuous map such that:
  \begin{equation*}
    \forall s \in [0,1]: H(s, 0) = \gamma_1(s) \: \land \: H(s, 1) = \gamma_2(s)
  \end{equation*}

  Now let $\gamma_1, \gamma_2 \in C([0, 1], Y)$ such that $\gamma_1(0) = \gamma_2(0) \: \land \: \gamma_1(1) = \gamma_2(1)$ i.e. that paths start and end at the same point.
  The homotopy $H \in C([0,1]^2, Y)$ is called \textbf{endpoint preserving} if it is a free homotopy between $\gamma_1$ and $\gamma_2$ and additionally it should have the following property: 
  \begin{equation*}
    \forall t\in[0,1]\forall i\in \{1,2\}: H(0, t) = \gamma_i(0) \: \land \: H(1, t) = \gamma_i(1)
  \end{equation*}

  If a free homotopy $H$ between the two paths $\gamma_1$ and $\gamma_2$ exists they are called \textbf{freely homotopic} $\gamma_1 \overset{\cdot}{\sim}_H \gamma_2$.
  If the two paths $\gamma_1$ and $\gamma_2$ have the same start and end point and there exists a endpoint preserving homotopy $H$ between them then they are called 
  \textbf{homotopic} $\gamma_1 \sim_H \gamma_2$. If a path is (freely) homotopic to a constant path ($[0,1] \to Y, t \mapsto y \in Y$) then it is called \textbf{null-homotopic}.

  Lastly define the homotpy relation $\sim$ on $C([0,1], Y)$ where 
  \begin{equation*}
    \gamma_1, \gamma_2 \in C([0,1], Y): \gamma_1 \sim \gamma_2 \iff \exists H \text{ homotopy between } \gamma_1 \text{ and } \gamma_2.
  \end{equation*}
\end{defin}

In this thesis all homotopies are endpoint preserving if it is not explicitly stated that they should be free homotopies.

\begin{defin} \label{def:connectedness}
  \begin{itemize}
    \item[] % this has to be here because of the newline after the defition label
    \item $Y$ \textbf{connected} $\;\longeq\; \exists A,B \in \T_Y\setminus \{\emptyset\}: A \cap B = \emptyset \: \land \: Y = A \cup B \Rightarrow A = \emptyset \: \lor \: B = \emptyset$
    \item $Y$ \textbf{path-connected} $\;\longeq\; \forall x,y \in Y \exists\gamma \in C([0,1], Y): \gamma(0) = x \: \land \: \gamma(1) = y$
    \item $Y$ \textbf{locally (path-)connected} $\;\longeq\; \forall y \in Y\forall U \in \NB_y\exists V\in \NB_y: V \subseteq U \: \land \: V$ (path-)connected
  \end{itemize}
\end{defin}

The following definitions only work in metric spaces:

\begin{defin}
  \begin{itemize}
    \item[] % this has to be here because of the newline after the defition label
    \item $X$ \textbf{uniformly locally path-connected} (u.l.p.c.) $\: \longeq \\ \forall \varepsilon \in \R_{>0} \exists \delta \in \R_{>0}\forall x \in X: \forall x_1, x_2 \in B_{\delta}(x)\exists \gamma \in C([0,1], B_{\varepsilon}(x)): \gamma(0) = x_1 \: \land \: \gamma(1) = x_2$
    \item $X$ \textbf{semi-locally simply connected} (s.l.s.c) $\: \longeq \\ \forall x\in X\exists \varepsilon \in \R_{>0}$: Every loop in $B_{\varepsilon}(x)$ is null-homotopic in $X$
    \item $X$ \textbf{uniformly semi-locally simply connected} (u.s.l.s.c) $\: \longeq \\ \exists \varepsilon \in \R_{>0}\forall x\in X$: Every loop in $B_{\varepsilon}(x)$ is null-homotopic in $X$
  \end{itemize}
  \vspace*{10pt}
  For clarification, the difference between semi-locally simply connectedness and uniformly semi-locally simply connected is 
  that in the first case $\varepsilon$ is choosen dependend on $x$ and in the latter case it is chosen independently for all $x$.
\end{defin}

\begin{lemma} \label{lem:homotopy-equivalence}
  The homotopy relation $\sim$ of paths is an equivalence relation on the set $C([0,1],Y)$.
\end{lemma}

\begin{proof}
  Let $\gamma_1, \gamma_2, \gamma_3 \in C([0,1],Y)$ with $\gamma_1(0) = \gamma_2(0) = \gamma_3(0)$ and $\gamma_1(1) = \gamma_2(1) = \gamma_3(1)$.
  
  \textit{Reflexivity:}
  Consinder the homotopy $H: [0,1]^2 \to Y, \: (s, t) \mapsto \gamma_1(s)$. 
  
  It holds that $\forall s\in [0,1]: H(s, 0) = \gamma_1(s),\: H(s, 1) = \gamma_1(s)$. And thus $\gamma_1 \sim_H \gamma_1 \Rightarrow \gamma_1 \sim \gamma_1$.

  \textit{Symmetry:} 
  Assume that $\gamma_1 \sim \gamma_2$. This means there exists a homotopy $H$ such that $\gamma_1 \sim_H \gamma_2$. 
  Now define the homotopy $F: [0,1]^2 \to Y, \: (s, t) \mapsto H(s, 1-t)$. 
  
  From this definition it follows that $\forall s \in [0,1]:$
  \begin{align*}
    F(s, 0) &= H(s, 1 - 0) = \gamma_2(s) \\
    F(s, 1) &= H(s, 1 - 1) = \gamma_1(s)
  \end{align*}
  and hence $\gamma_2 \sim_F \gamma_1 \Rightarrow \gamma_2 \sim \gamma_1$.

  \textit{Transitivity:}
  Assume that $\gamma_1 \sim \gamma_2$ and $\gamma_2 \sim \gamma_3$. Let $H$ be a homotopy between $\gamma_1$ and $\gamma_2$ and let $F$ be a homotopy between $\gamma_2$ and $\gamma_3$. 
  Define the homotopy
  \begin{equation*} 
    G: [0,1]^2 \to Y, \: (s,t) \mapsto \begin{cases}
      H(s, 2t), &t \in [0, {1 \over 2}], \\
      F(s, 2t - 1), &t \in [{1 \over 2}, 1].
    \end{cases}
  \end{equation*}

  Then the following equations hold:
  \begin{align*}
    G(s,0) &= H(s,0) = \gamma_1(s), \\
    G\left(s,{1 \over 2}\right) &= H(s, 1) = \gamma_2(s) = F(s, 0), \\
    G(s, 1) &= F(s, 1) = \gamma_3(s).
  \end{align*}
  and thus $\gamma_1 \sim_G \gamma_3 \Rightarrow \gamma_1 \sim \gamma_3$.
\end{proof}

\begin{rem} \label{rem:reparam}
  Let $\alpha: [0,1] \to [0,1]$ be a continuous map with $\alpha(0) = 0, \alpha(1) = 1$ and let $\gamma \in C([0,1], Y)$. Then it follows that $\gamma \sim \gamma \circ \alpha$.
\end{rem}

\begin{proof}
  Let $F: [0,1]^2 \to Y, (s, t) \mapsto \gamma((1 - t)s + t\alpha(s))$.

  At first fix $s \in [0,1]$ then $s, \alpha(s) \in [0,1]$ and $(1 - t)s + t\alpha(s)$ is a convex combination of the two points. 
  $[0,1]$ is a convex set and thus $\forall t\in[0,1]: (1 - t)s + t\alpha(s) \in [0,1]$. Because of the continuity of the operations $+$ and $\cdot$ in $[0,1]$ 
  and the continuity of $\alpha$ the map $F$ is continuous and the following hold:
  \begin{align*}
    F(s, 0) &= \gamma(s), \\
    F(s, 1) &= \gamma(\alpha(s)),
  \end{align*}
  and thus $F$ is a homotopy between $\gamma \sim \gamma \circ \alpha$.
\end{proof}

\begin{defin}
  Let $y_0 \in Y$. The \textbf{fundamental group} of the topological space $Y$ with basepoint $y_0$ is defined as follows:
  \begin{equation*}
    \pi_1(Y, y_0) := (\{\gamma \in C([0,1],Y) \: | \: \gamma(0) = \gamma(1) = y_0\}/_{\sim}, \: *)
  \end{equation*}
  where 
  \begin{equation*} 
    *: C([0,1], Y) \times C([0,1], Y) \to C([0,1], Y), (\gamma_1, \gamma_2) \mapsto \left(t \mapsto \begin{cases}
    \gamma_1(2t),   &t \in [0, {1 \over 2}], \\
    \gamma_2(2t-1), &t \in [{1 \over 2}, 1],
  \end{cases}\right)
\end{equation*}
is the concatenation of paths. The case $t = {1\over2}$ is included in both cases because the operation is only defined for pahts $\gamma_1, \gamma_2$ if $\gamma_1(1) = \gamma_2(0)$. 
In the case of the fundamental group this is fulfilled for every path because all paths are loops that start and end at the same basepoint.

The operation $*: \pi_1(Y, y_0) \times \pi_1(Y, y_0) \to \pi_1(Y, y_0)$ is defined as follows:
\begin{equation*}
  ([\gamma_1]_{\sim}, [\gamma_2]_{\sim}) \mapsto [\gamma_1]_{\sim} * [\gamma_2]_{\sim} = [\gamma_1 * \gamma_2]_{\sim}
\end{equation*} 
\end{defin}

\begin{thm}
  Let $y_0 \in Y$. The fundamental group $\pi_1(Y, y_0)$ is a well-defined group. 
\end{thm}

% TODO: pictures of reparams
\begin{proof}
  By lemma \ref{lem:homotopy-equivalence} $\sim$ is a well-defined equivalence relation and thus the quotient in the definition is well-defined and the set of loops based at $y_0$ is decomposed into equivalence classes.
  Thus it remains to be shown that the set of equivalence classes of loops up to homotopy together with the concatenation operation $*$ fulfills the group axioms.

  Therefore let $\gamma_1, \gamma_2$ be loops in $Y$ based at $y_0 \in Y$. From the definition of $*$ it is clear that $\gamma_1 * \gamma_2$ is again a loop in $Y$ based at $y_0$.
  Now let $\tilde{\gamma_1}$ be another loop based at $y_0$ that is homotopic to $\gamma_1$ via the homotopy $H$. Define $F: [0,1]^2 \to Y$ as follows:
  \begin{equation*}
    (s, t) \mapsto \begin{cases}
      H(2s, t), &s \in [0,{1 \over 2}], \\
      \gamma_2(s), &s \in [{1 \over 2}, 1].
    \end{cases}
  \end{equation*}
  This is a well-defined homotopy between $\gamma_1 * \gamma_2$ and $\tilde{\gamma_1} * \gamma_2$ because of:
  \begin{align*}
    F(s, 0) &= \gamma_1 * \gamma_2, \\
    F(s, 1) &= \tilde{\gamma_1} * \gamma_2,
  \end{align*}
  There is no problem for $t = {1 \over 2}$ because the homotopy is endpoint preserving. Thus it holds that $\gamma_1 \sim \tilde{\gamma_1} \Rightarrow \gamma_1 * \gamma_2 \sim \tilde{\gamma_1} * \gamma_2$.
  And a similiar argument shows $\gamma_1 \sim \tilde{\gamma_1} \Rightarrow \gamma_2 * \gamma_1 \sim  \gamma_2 * \tilde{\gamma_1}$. 
  This proofs that the group operations is not dependend on the choice of representative of the particular equivalence class which means that the group operation is well-defined.

  Now let $\gamma_3$ be another loop in $Y$ based at $y_0$. 
  
  Consider the map $\alpha: [0,1] \to [0,1]$ which is defined as:
  \begin{equation*}
    t \mapsto \begin{cases}
      2t, &t \in [0, {1 \over 4}], \\
      t + {1 \over 4}, &t \in [{1\over4}, {1\over2}], \\
      {1\over2}t + {1\over2}, &t\in[{1\over2}, 1],
    \end{cases}
  \end{equation*}
  
  then $\alpha(0) = 0, \alpha(1) = 1$ and $((\gamma_1 * \gamma_2) * \gamma_3) \circ \alpha = \gamma_1 * (\gamma_2 * \gamma_3)$ and thus $(\gamma_1 * \gamma_2) * \gamma_3 \sim \gamma_1 * (\gamma_2 * \gamma_3)$ by Remark \ref{rem:reparam}.
  Hence the operation $*$ is associative.

  Next consider the constant path $e_{y_0}: [0,1] \to Y, \: t \mapsto y_0$ at $y_0$. 
  
  \vspace*{5pt}
  Define $\alpha_1: [0,1] \to [0,1], \: t \mapsto \begin{cases}
    0, &t\in[0, {1\over2}],\\
    2t, &t\in[{1\over2}, 1],
  \end{cases}$ and $\alpha_2: [0,1]\to [0,1], \: t \mapsto \begin{cases}
    2t, &t\in[0, {1\over2}],\\
    0, &t\in[{1\over2}, 1],
  \end{cases}$
  \vspace*{5pt}
  with $\alpha_1(0) = \alpha_2(0) = 0$ and $\alpha_1(1) = \alpha_2(1) = 1$.

  It follows for a loop $\gamma$ in $Y$ based at $y_0$ that: $e_{y_0} * \gamma = \gamma \circ \alpha_1$ and $\gamma * e_{y_0} = \gamma \circ \alpha_2$ and thus by Remark \ref{rem:reparam}:
  \begin{equation*}
    \gamma * e_{y_0} \sim \gamma \sim e_{y_0} * \gamma
  \end{equation*}
  This means that $e_{y_0}$ is the identity element with respect to $*$.

  Now let $\gamma^-$ be a loop in $Y$ based at $y_0$ with $\forall t \in [0,1]: \gamma^-(t) = \gamma(1 - t)$ and consider the homotopy $G: [0,1]^2 \to Y$ with 
  \begin{equation*}
    (s, t) \mapsto \begin{cases}
      \gamma(2s(1-t)), &t \in [0, {1\over2}], \\
      \gamma(2(1-s)(1-t)), &t \in [{1\over2}, 1] 
    \end{cases}
  \end{equation*}
  is a homotopy between $\gamma * \gamma^-$ and $e_{y_0}$ which means $[\gamma] * [\gamma^-] = [\gamma * \gamma^-] = [e_{y_0}]$ and thus $[\gamma^-] = [\gamma]^{-1}$. 
\end{proof}

\section{Discretization of the fundamental group}

\section{Inverse Limit}

\section{Counterexamples}

\clearpage
\bibliographystyle{alpha}
\bibliography{ref} % see references.bib for bibliography management

\end{document}