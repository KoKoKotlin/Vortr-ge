\documentclass[a4paper, 11pt, twoside]{article}
\usepackage[outer={2.5cm},inner={3cm},tmargin={2.5cm},bmargin={2cm}]{geometry}
\usepackage{amsmath}
\usepackage{amssymb}
\usepackage{amsthm}
\usepackage{hyperref}
\usepackage{colonequals}
\usepackage{xfrac}
\usepackage{tikz}
\usepackage{tikz-cd}

\usepackage{graphicx}
\usepackage{caption}

\setlength{\parindent}{0mm}
\setcounter{MaxMatrixCols}{100}

\newcommand{\R}[0]{\mathbb{R}}
\newcommand{\N}[0]{\mathbb{N}}
\newcommand{\T}[0]{\mathcal{T}}
\newcommand{\NB}[0]{\mathcal{N}}
\newcommand*{\logeq}{\ratio\Leftrightarrow}
\newcommand*{\longeq}{\ratio\Longleftrightarrow}
\newcommand*{\bigdot}{\mathpalette\bigcdot@{.5}}
\DeclareRobustCommand{\loongrightarrow}{%
  \DOTSB\relbar\joinrel\relbar\joinrel\rightarrow
}

\title{Seminar thesis - Fundamental groups as limits of discrete fundametal groups}
\author{Yannik Höll}
\date{\today}

\newtheoremstyle{break}%
{7pt}{7pt}%
{}{}%
{\bfseries}{:}% % Note that final punctuation is omitted.
{\newline}{}

\theoremstyle{break}
\newtheorem{thm}{Theorem}[section]

\theoremstyle{break}
\newtheorem{defin}[thm]{Definition}
\newtheorem{rem}[thm]{Remark}
\newtheorem{lemma}[thm]{Lemma}

\begin{document}

\nocite{*}

\maketitle

% TODO: add introduction with a summary of the main questions of the paper 
% i.e. can the fundamental group be recovered as a limit of discrete fundametal groups 

\section{Basic topological definitions} \label{section-basic-defs}

% TODO: only some defins, ref to def of topology, topological space, etc. 

\textbf{Notation:}
\begin{enumerate}
  \item Let $Y$ be a topological space with the topology $\T_Y$,
  \item Let $X$ be a metric space with the metric $d_X$ which is also a topological space with the topology induced by its metric,
  \item If $Z_1, Z_2$ are topological spaces then $C(Z_1, Z_2)$ is the set of all continous maps from $Z_1$ to $Z_2$,
  \item Let $y \in Y$ then $\NB_y$ is the set of all neighbourhoods of the element $y$.
  \item Let $\varepsilon \in \R_{>0}$ and let $x \in X$: $B_{\varepsilon}(x) := \{ y\in X \: | \: d_X(x, y) < \varepsilon\}$
\end{enumerate}

\begin{defin}
  A \textbf{path} $\gamma$ in $Y$ is a continuous map $\R \supseteq [0,1] \to Y$.
\end{defin}

\begin{defin}
  Let $\gamma_1, \gamma_2 \in C([0, 1], Y)$. A \textbf{free homotopy} $H: [0,1]^2 \to Y$ between $\gamma_1$ and $\gamma_2$ is a continuous map such that:
  \begin{equation*}
    \forall s \in [0,1]: H(s, 0) = \gamma_1(s) \: \land \: H(s, 1) = \gamma_2(s)
  \end{equation*}

  Now let $\gamma_1, \gamma_2 \in C([0, 1], Y)$ such that $\gamma_1(0) = \gamma_2(0) \: \land \: \gamma_1(1) = \gamma_2(1)$ i.e. that paths start and end at the same point.
  The homotopy $H \in C([0,1]^2, Y)$ is called \textbf{endpoint preserving} if it is a free homotopy between $\gamma_1$ and $\gamma_2$ and additionally it should have the following property: 
  \begin{equation*}
    \forall t\in[0,1]\forall i\in \{1,2\}: H(0, t) = \gamma_i(0) \: \land \: H(1, t) = \gamma_i(1)
  \end{equation*}

  If a free homotopy $H$ between the two paths $\gamma_1$ and $\gamma_2$ exists they are called \textbf{freely homotopic} $\gamma_1 \overset{\cdot}{\sim}_H \gamma_2$.
  If the two paths $\gamma_1$ and $\gamma_2$ have the same start and end point and there exists a endpoint preserving homotopy $H$ between them then they are called 
  \textbf{homotopic} $\gamma_1 \sim_H \gamma_2$. If a path is (freely) homotopic to a constant path ($[0,1] \to Y, t \mapsto y \in Y$) then it is called \textbf{null-homotopic}.

  Lastly define the homotpy relation $\sim$ on $C([0,1], Y)$ where 
  \begin{equation*}
    \gamma_1, \gamma_2 \in C([0,1], Y): \gamma_1 \sim \gamma_2 \iff \exists H \text{ homotopy between } \gamma_1 \text{ and } \gamma_2.
  \end{equation*}
\end{defin}

In this thesis all homotopies are endpoint preserving if it is not explicitly stated that they should be free homotopies.

\begin{defin} \label{def:connectedness}
  \begin{itemize}
    \item[] % this has to be here because of the newline after the defition label
    \item $Y$ \textbf{connected} $\;\longeq\; \exists A,B \in \T_Y\setminus \{\emptyset\}: A \cap B = \emptyset \: \land \: Y = A \cup B \Rightarrow A = \emptyset \: \lor \: B = \emptyset$
    \item $Y$ \textbf{path-connected} $\;\longeq\; \forall x,y \in Y \exists\gamma \in C([0,1], Y): \gamma(0) = x \: \land \: \gamma(1) = y$
    \item $Y$ \textbf{locally (path-)connected} $\;\longeq\; \forall y \in Y\forall U \in \NB_y\exists V\in \NB_y: V \subseteq U \: \land \: V$ (path-)connected
  \end{itemize}
\end{defin}

The following definitions only work in metric spaces:

\begin{defin}
  \begin{itemize}
    \item[] % this has to be here because of the newline after the defition label
    \item $X$ \textbf{uniformly locally path-connected} (u.l.p.c.) $\: \longeq \\ \forall \varepsilon \in \R_{>0} \exists \delta \in \R_{>0}\forall x \in X: \forall x_1, x_2 \in B_{\delta}(x)\exists \gamma \in C([0,1], B_{\varepsilon}(x)): \gamma(0) = x_1 \: \land \: \gamma(1) = x_2$
    \item $X$ \textbf{semi-locally simply connected} (s.l.s.c) $\: \longeq \\ \forall x\in X\exists \varepsilon \in \R_{>0}$: Every loop in $B_{\varepsilon}(x)$ is null-homotopic in $X$
    \item $X$ \textbf{uniformly semi-locally simply connected} (u.s.l.s.c) $\: \longeq \\ \exists \varepsilon \in \R_{>0}\forall x\in X$: Every loop in $B_{\varepsilon}(x)$ is null-homotopic in $X$
  \end{itemize}
  \vspace*{10pt}
  For clarification, the difference between semi-locally simply connectedness and uniformly semi-locally simply connected is 
  that in the first case $\varepsilon$ is choosen dependend on $x$ and in the latter case it is chosen independently for all $x$.
\end{defin}

\begin{lemma} \label{lem:homotopy-equivalence}
  The homotopy relation $\sim$ of paths is an equivalence relation on the set $C([0,1],Y)$.
\end{lemma}

\begin{proof}
  Let $\gamma_1, \gamma_2, \gamma_3 \in C([0,1],Y)$ with $\gamma_1(0) = \gamma_2(0) = \gamma_3(0)$ and $\gamma_1(1) = \gamma_2(1) = \gamma_3(1)$.
  
  \textit{Reflexivity:}
  Consinder the homotopy $H: [0,1]^2 \to Y, \: (s, t) \mapsto \gamma_1(s)$. 
  
  It holds that $\forall s\in [0,1]: H(s, 0) = \gamma_1(s),\: H(s, 1) = \gamma_1(s)$. And thus $\gamma_1 \sim_H \gamma_1 \Rightarrow \gamma_1 \sim \gamma_1$.

  \textit{Symmetry:} 
  Assume that $\gamma_1 \sim \gamma_2$. This means there exists a homotopy $H$ such that $\gamma_1 \sim_H \gamma_2$. 
  Now define the homotopy $F: [0,1]^2 \to Y, \: (s, t) \mapsto H(s, 1-t)$. 
  
  From this definition it follows that $\forall s \in [0,1]:$
  \begin{align*}
    F(s, 0) &= H(s, 1 - 0) = \gamma_2(s) \\
    F(s, 1) &= H(s, 1 - 1) = \gamma_1(s)
  \end{align*}
  and hence $\gamma_2 \sim_F \gamma_1 \Rightarrow \gamma_2 \sim \gamma_1$.

  \textit{Transitivity:}
  Assume that $\gamma_1 \sim \gamma_2$ and $\gamma_2 \sim \gamma_3$. Let $H$ be a homotopy between $\gamma_1$ and $\gamma_2$ and let $F$ be a homotopy between $\gamma_2$ and $\gamma_3$. 
  Define the homotopy
  \begin{equation*} 
    G: [0,1]^2 \to Y, \: (s,t) \mapsto \begin{cases}
      H(s, 2t), &t \in [0, {1 \over 2}], \\
      F(s, 2t - 1), &t \in [{1 \over 2}, 1].
    \end{cases}
  \end{equation*}

  Then the following equations hold:
  \begin{align*}
    G(s,0) &= H(s,0) = \gamma_1(s), \\
    G\left(s,{1 \over 2}\right) &= H(s, 1) = \gamma_2(s) = F(s, 0), \\
    G(s, 1) &= F(s, 1) = \gamma_3(s).
  \end{align*}
  and thus $\gamma_1 \sim_G \gamma_3 \Rightarrow \gamma_1 \sim \gamma_3$.
\end{proof}

\begin{rem} \label{rem:reparam}
  Let $\alpha: [0,1] \to [0,1]$ be a continuous map with $\alpha(0) = 0, \alpha(1) = 1$ and let $\gamma \in C([0,1], Y)$. Then it follows that $\gamma \sim \gamma \circ \alpha$.
\end{rem}

\begin{proof}
  Let $F: [0,1]^2 \to Y, (s, t) \mapsto \gamma((1 - t)s + t\alpha(s))$.

  At first fix $s \in [0,1]$ then $s, \alpha(s) \in [0,1]$ and $(1 - t)s + t\alpha(s)$ is a convex combination of the two points. 
  $[0,1]$ is a convex set and thus $\forall t\in[0,1]: (1 - t)s + t\alpha(s) \in [0,1]$. Because of the continuity of the operations $+$ and $\cdot$ in $[0,1]$ 
  and the continuity of $\alpha$ the map $F$ is continuous and the following hold:
  \begin{align*}
    F(s, 0) &= \gamma(s), \\
    F(s, 1) &= \gamma(\alpha(s)),
  \end{align*}
  and thus $F$ is a homotopy between $\gamma \sim \gamma \circ \alpha$.
\end{proof}

\begin{defin}
  Let $y_0 \in Y$. The \textbf{fundamental group} of the topological space $Y$ with basepoint $y_0$ is defined as follows:
  \begin{equation*}
    \pi_1(Y, y_0) := (\{\gamma \in C([0,1],Y) \: | \: \gamma(0) = \gamma(1) = y_0\}/_{\sim}, \: *)
  \end{equation*}
  where 
  \begin{equation*} 
    *: C([0,1], Y) \times C([0,1], Y) \to C([0,1], Y), (\gamma_1, \gamma_2) \mapsto \left(t \mapsto \begin{cases}
    \gamma_1(2t),   &t \in [0, {1 \over 2}], \\
    \gamma_2(2t-1), &t \in [{1 \over 2}, 1],
  \end{cases}\right)
\end{equation*}
is the concatenation of paths. The case $t = {1\over2}$ is included in both cases because the operation is only defined for pahts $\gamma_1, \gamma_2$ if $\gamma_1(1) = \gamma_2(0)$. 
In the case of the fundamental group this is fulfilled for every path because all paths are loops that start and end at the same basepoint.

The operation $*: \pi_1(Y, y_0) \times \pi_1(Y, y_0) \to \pi_1(Y, y_0)$ is defined as follows:
\begin{equation*}
  ([\gamma_1]_{\sim}, [\gamma_2]_{\sim}) \mapsto [\gamma_1]_{\sim} * [\gamma_2]_{\sim} = [\gamma_1 * \gamma_2]_{\sim}
\end{equation*} 
\end{defin}

\begin{thm}
  Let $y_0 \in Y$. The fundamental group $\pi_1(Y, y_0)$ is a well-defined group. 
\end{thm}

\begin{proof}
  By lemma \ref{lem:homotopy-equivalence} $\sim$ is a well-defined equivalence relation and thus the quotient in the definition is well-defined and the set of loops based at $y_0$ is decomposed into equivalence classes.
  Thus it remains to be shown that the set of equivalence classes of loops up to homotopy together with the concatenation operation $*$ fulfills the group axioms.

  Therefore let $\gamma_1, \gamma_2$ be loops in $Y$ based at $y_0 \in Y$. From the definition of $*$ it is clear that $\gamma_1 * \gamma_2$ is again a loop in $Y$ based at $y_0$.
  Now let $\tilde{\gamma_1}$ be another loop based at $y_0$ that is homotopic to $\gamma_1$ via the homotopy $H$. Define $F: [0,1]^2 \to Y$ as follows:
  \begin{equation*}
    (s, t) \mapsto \begin{cases}
      H(2s, t), &s \in [0,{1 \over 2}], \\
      \gamma_2(s), &s \in [{1 \over 2}, 1].
    \end{cases}
  \end{equation*}
  This is a well-defined homotopy between $\gamma_1 * \gamma_2$ and $\tilde{\gamma_1} * \gamma_2$ because of:
  \begin{align*}
    F(s, 0) &= \gamma_1 * \gamma_2, \\
    F(s, 1) &= \tilde{\gamma_1} * \gamma_2,
  \end{align*}
  
  There is no problem for $t = {1 \over 2}$ because the homotopy is endpoint preserving. Thus it holds that $\gamma_1 \sim \tilde{\gamma_1} \Rightarrow \gamma_1 * \gamma_2 \sim \tilde{\gamma_1} * \gamma_2$.
  And a similiar argument shows $\gamma_1 \sim \tilde{\gamma_1} \Rightarrow \gamma_2 * \gamma_1 \sim  \gamma_2 * \tilde{\gamma_1}$. 
  This proofs that the group operations is not dependend on the choice of representative of the particular equivalence class which means that the group operation is well-defined.

  Now let $\gamma_3$ be another loop in $Y$ based at $y_0$. 

  Consider the map $\alpha: [0,1] \to [0,1]$ which is defined as:
  \begin{equation*}
    t \mapsto \begin{cases}
      2t, &t \in [0, {1 \over 4}], \\
      t + {1 \over 4}, &t \in [{1\over4}, {1\over2}], \\
      {1\over2}t + {1\over2}, &t\in[{1\over2}, 1],
    \end{cases}
  \end{equation*}
  then $\alpha(0) = 0, \alpha(1) = 1$ and $((\gamma_1 * \gamma_2) * \gamma_3) \circ \alpha = \gamma_1 * (\gamma_2 * \gamma_3)$ and thus $(\gamma_1 * \gamma_2) * \gamma_3 \sim \gamma_1 * (\gamma_2 * \gamma_3)$ by Remark \ref{rem:reparam}.
  Hence the operation $*$ is associative.
  

  Next consider the constant path $e_{y_0}: [0,1] \to Y, \: t \mapsto y_0$ at $y_0$. 
  
  \vspace*{5pt}
  Define $\alpha_1: [0,1] \to [0,1], \: t \mapsto \begin{cases}
    0, &t\in[0, {1\over2}],\\
    2t, &t\in[{1\over2}, 1],
  \end{cases}$ and $\alpha_2: [0,1]\to [0,1], \: t \mapsto \begin{cases}
    2t, &t\in[0, {1\over2}],\\
    0, &t\in[{1\over2}, 1],
  \end{cases}$
  \vspace*{5pt}
  with $\alpha_1(0) = \alpha_2(0) = 0$ and $\alpha_1(1) = \alpha_2(1) = 1$.

  It follows for a loop $\gamma$ in $Y$ based at $y_0$ that: $e_{y_0} * \gamma = \gamma \circ \alpha_1$ and $\gamma * e_{y_0} = \gamma \circ \alpha_2$ and thus by Remark \ref{rem:reparam}:
  \begin{equation*}
    \gamma * e_{y_0} \sim \gamma \sim e_{y_0} * \gamma
  \end{equation*}
  This means that $e_{y_0}$ is the identity element with respect to $*$.

  Now let $\gamma^-$ be a loop in $Y$ based at $y_0$ with $\forall t \in [0,1]: \gamma^-(t) = \gamma(1 - t)$ and consider the homotopy $G: [0,1]^2 \to Y$ with 
  \begin{equation*}
    (s, t) \mapsto \begin{cases}
      \gamma(2s(1-t)), &t \in [0, {1\over2}], \\
      \gamma(2(1-s)(1-t)), &t \in [{1\over2}, 1] 
    \end{cases}
  \end{equation*}
  is a homotopy between $\gamma * \gamma^-$ and $e_{y_0}$ which means $[\gamma] * [\gamma^-] = [\gamma * \gamma^-] = [e_{y_0}]$ and thus $[\gamma^-] = [\gamma]^{-1}$. 
\end{proof}

\section{Discretization of the fundamental group}
\textbf{Notation:}
\begin{enumerate}
  \item Let $n \in \N$: $[n] := \{0, 1, \cdots, n\}$,
  \item Let $n,m \in \N$ with $n \leq m$: $[n,m] := \{n, n+1, \cdots, m\}$.
\end{enumerate}

The notion of the fundamental group introduced in the previous chapter is very a useful tool for detemining if two topological spaces are \textbf{not} homeomorphic.
The fundamental group has this ability because it is a functor from the category of topological spaces with a fixed basepoint to the category of groups. 
And if the images of a functor are not isomorphic than the arguments that mapped to these images can not be isomorphic in their category because functors preserve isomorphims.

A major disadvantage of the fundamental group is that it is very computationally difficult to calculate.
Thats the reason why there exist many other branches of algebraic topology concerned with other topological invariants that are invariant under homeomorphisms. 
Another fruitful idea would be to reduce the difficulty in calculating the fundamental group by approximating it via discrete fundamental groups. 
This concept it explored further in the following chapter.

\begin{defin} \label{def:discrete-path}
  Let $\theta \in \R_{>0}$ and $n \in \N$. A \textbf{discrete path} ($\theta$-path) with length $n$ is a map $Z: \R \supseteq [n] \to X$ that is $\theta$-lipschitz, i.e. for $\forall n_1, n_2 \in [n]$:
  \begin{equation}
    d_X(Z(n_2), Z(n_1)) \leq \theta |n_2 - n_1|
  \end{equation}
  An important special case is $n_2 = n_1 + 1$ for $n_1, n_2 \in [n]$ for which holds: $d_X(Z(n_1), Z(n_2)) \leq \theta$. This motivates a second
  definition if the discrete path namely as a sequence of points $(z_i)_{i=0}^n$ with $z_i \in X$ and for $\forall i \in \{0, 1, \cdots n-1\}$:
  \begin{equation}
    d_X(z_i, z_{i+1}) \leq \theta
  \end{equation}

  Let $m \in \N$ with $m \geq n$ and let $Z': [m] \to X$ be a discrete path. $Z'$ is called a \textbf{lazification} of $Z$ if there exists a surjective, monotone map 
  $f: [m] \to [n]$ such that $Z' = Z \circ f$.
\end{defin}

% Better definition of homotopy in BCW14
\begin{defin}
  Let $\theta \in \R_{>0}$, $n,m \in \N$ and let $Z_1: [n] \to X, Z_2: [n] \to X$ be two discrete $\theta$-paths in $X$. 
  A \textbf{(free) $\theta$-grid homotopy} is a $\theta$-lipschitz map $H: [n] \times [m] \to X$ such that:
  \begin{equation*}
    H(\cdot, 0) = Z_1 \: \land \: H(\cdot, m) = Z_2
  \end{equation*}
  where the product $[n] \times [m]$ is a metric space with the $\ell_1$ metric\footnote{i.e. for $x = (x_1, x_2), y = (y_1, y_2) \in [n] \times [m]: d_{\ell_1}(x, y) = |x_1 - y_1| + |x_2 - y_2|$}.

  A free $\theta$-grid homotopy is a $\theta$-grid homotopy if it preserves the endpoints of the discrete paths i.e.: $\forall t\in [m]$:
  \begin{equation*}
    H(0,t) = Z_1(0) = Z_2(0) \: \land \: H(n,t) = Z_1(n) = Z_2(n)
  \end{equation*}
  for paths $Z_1, Z_2$ with the same start and end points. 
\end{defin}

% source BCW14
\begin{rem}
  Let $(x_0, x_1, \cdots x_n = x_0)$ and $(y_0, y_1, \cdots y_n = y_0)$ with $n \in \N$ be two closed $\theta$-paths in $X$. 
  One can imagine a $\theta$-grid homotopy as the following grid:
  \begin{equation*}
    \begin{matrix}
      x_0 & x_1 & x_2 & \cdots & x_n \\
      z_0^1 & z_1^1 & z_2^1 & \cdots & z_n^1 \\
      \vdots & \vdots & \vdots & \vdots & \vdots \\
      z_0^{t} & z_1^{t} & z_2^{t} & \cdots & z_n^{t} \\
      y_0 & y_1 & y_2 & \cdots & y_n \\
    \end{matrix}
  \end{equation*}
  where the rows are $\theta$-loops in $X$ and the columns are $\theta$-paths in $X$ and $t\in \N$.
\end{rem}

\begin{defin}
  Let $\theta \in \R_{>0}$ and let $Z_1, Z_2$ be discrete $\theta$-paths. $Z_1$ and $Z_2$ are said to be $\theta$-homotopic 
  if there exist lazifications $Z_1'$ of $Z_1$ and $Z_2'$ of $Z_2$ with the same length and a $\theta$-grid homotopy between $Z_1'$ and $Z_2'$.
  
  This is denoted as $Z_1 \sim_{\theta} Z_2$.
\end{defin}

\begin{defin}
  Let $\theta \in \R_{>0}$ and $x_0 \in X$. $\mathcal{C}_{\theta}(X, x_0)$ is the set of all discrete closed $\theta$-paths in $X$ starting and ending at $x_0$. Now take $x, y \in \mathcal{C}_{\theta}(X, x_0)$.
  
  By the definition \ref{def:discrete-path} there are $m,n \in \N$ such that we can write: 
  \begin{equation*}
    x = (x_0, x_1, \cdots x_n = x_0), \: y = (y_0, y_1, \cdots y_m = y_0)
  \end{equation*} where $\forall i \in [n]: x_i \in X$ and $\forall i \in [m]: y_i \in X$.

  Now define the concatenation of paths $*: \mathcal{C}_{\theta}(X, x_0) \times \mathcal{C}_{\theta}(X, x_0) \to \mathcal{C}_{\theta}(X, x_0), (x, y) \to x * y$ where
  \begin{equation*}
    x * y = (x_0, x_1, \cdots, x_n = y_0, y_1, \cdots, y_m)
  \end{equation*}
  i.e. $x * y: [n+m] \to X, t \mapsto \begin{cases}
    x_t, &t \in [n], \\
    y_{t-n}, &t \in \{n, \cdots, n+m\}
  \end{cases}$.
\end{defin}

\begin{lemma} \label{lem:discrete-homotopy}
  Let $\theta \in \R_{>0}$ then the $\theta$-grid homotopy relation $\sim_{\theta}$ on the set $\mathcal{C}_{\theta}(X, x_0)$ is a equivalence relation.
\end{lemma}

% TODO: look up proof in BCW14
\begin{proof}
  Similiar to proof of \ref{lem:homotopy-equivalence}.
\end{proof}

\begin{defin}
  Let $\theta \in \R_{>0}$ and let $x_0 \in X$. $\pi_{1,\theta}(X, x_0) := (\mathcal{C}_{\theta}(X, x_0)/_{\sim_{\theta}}, *)$ is called the \textbf{discrete fundamental group} as scale $\theta$,
  with $*: \pi_{1,\theta}(X, x_0) \times \pi_{1,\theta}(X, x_0) \to \pi_{1,\theta}(X, x_0), ([x]_{\theta}, [y]_{\theta}) \mapsto [x * y]_{\theta}$.
\end{defin}

\begin{thm}
  Let $\theta \in \R_{>0}$ and $x_0 \in X$. The discrete fundamental group $\pi_{1,\theta}(X, x_0)$ is a well-defined group.
\end{thm}

\begin{proof}
  Firstly the set $\mathcal{C}_{\theta}(X, x_0)/_{\sim_{\theta}}$ is well-defined because of lemma \ref{lem:discrete-homotopy}.

  Now let $[x]_{\theta}, [y]_{\theta} \in \pi_{1, \theta}(X, x_0)$. Let $x, \tilde{x} \in [x]_{\theta}$ and $y \in [y]_{\theta}$.
  Then there exist lazifications $x', \tilde{x}'$ of $x, \tilde{x}$ and $y'$ of $y$ with the same length $m \in \N$ 
  and a $\theta$-grid homotopy $H: [n] \times [m] \to X$ with  between $x'$ and $\tilde{x}'$ ($n \in \N$).

  Let $k \in N$ and define the map $\tilde{H}: [n] \times [2m] \to X$ as follows:
  \begin{equation*}
    \tilde{H}(s, t) = \begin{cases}
       H(s, t), &t \in [m], \\
       y'(t-m), &t \in [m,2m].
    \end{cases}
  \end{equation*}
  Since the maps $H$ and $y'$ are $\theta$-lipschitz and $\forall s \in [n]: x_0 = H(s,m) = y(0)$ 
  it follows from the pasting lemma for $\theta$-lipschitz maps (\cite[Thm. 1]{kvalheim2021pasting}) that $\tilde{H}$ is $\theta$-lipschitz.
  Since $\forall t \in [2m]: \tilde{H}(0, t) = (x' * y')(t)$ and $\tilde{H}(n, t) = (\tilde{x}' * y')(t)$ it follows that $x * y \sim_{\theta} \tilde{x} * y$ and thus $[x * y]_{\theta} = [\tilde{x} * y]_\theta$.

  And similiar argument can be made for the right argument. 
  
  It follows that $[x * y]_{\theta} = [\tilde{x} * \tilde{y}]_{\theta}$ if $\tilde{y} \in [y]_\theta$ 
  which means that the group operation is independent from the choice of representative.

  \textit{Associativity:}
    Let $[x]_{\theta}, [y]_{\theta}, [z]_{\theta} \in \pi_{1,\theta}(X, x_0)$ 
    and let $x', y', z'$ be lazifications of the discrete paths with length $m$ which is the maximum of the lengths of $x,y,z$. Consider:
    \begin{align*}
      f(t) := (x' * y') * z' &= \begin{cases}
        (x' * y')(t), &t \in [2m], \\
        z'(t - 2m), &t \in [2m, 3m],
      \end{cases}\\
      g(t) := x' * (y' * z') &= \begin{cases}
        x'(t), &t \in [m], \\
        (y' * z')(t - m), &t \in [m, 3m]
      \end{cases}
    \end{align*}
  \begin{enumerate}
    \item $t \in [m]: f(t) = (x' * y')(t) = x'(t) = g(t)$
    \item $t \in [m,2m]: f(t) = (x' * y')(t) = y'(t - m) = (y' * z')(t - m) = g(t)$
    \item $t \in [2m,3m]: f(t) = z'(t - 2m) = (y' * z')(t - m) = g(t)$
  \end{enumerate}
  and thus $f = g\:$ i.e. $[x]_{\theta} * ([y]_{\theta} * [z]_{\theta}) = [x * (y * z)]_{\theta} = [(x * y) * z]_{\theta} = ([x]_{\theta} * [y]_{\theta}) * [z]_{\theta}$.
  
  \textit{Identity:}
  Let $e_{x_0}: [0] \to X, 0 \mapsto x_0$ be the constant path at $x_0$. Then $x * e_{x_0} = x'$ $e_{x_0} * x = x''$ with $x'$ and $x''$ lazifications of $x$ 
  and hence $e_{x_0} \sim_{\theta} x \sim_{\theta} e_{x_0}$.
  It follows that $[x]_{\theta} * [e_{x_0}]_{\theta} = [x * e_{x_0}]_{\theta} = [x]_{\theta}$ and $[e_{x_0}]_{\theta} * [x]_{\theta} = [e_{x_0} * x]_{\theta} = [x]_{\theta}$.
  
  \textit{Inverse:}
  Let $x \in \mathcal{C}_{\theta}(X, x_0)$ with length $m$. Define the path $x^-: [m] \to X, t \mapsto x(m - t)$ where $n$ is the length of $x$.

  Define $H: [m] \times [2m] \to X$ given by the following schema:
  \begin{equation*}
    \begin{matrix}
      x_0 & x_1 & x_2 & \cdots & x_{m-2} & x_{m-1} & x_m = x_0 & x_{m-1} & x_{m-2} & \cdots & x_1 & x_0 \\
      x_0 & x_1 & x_2 & \cdots & x_{m-2} & x_{m-1} & x_{m-1} & x_{m-1} & x_{m-2} & \cdots & x_1 & x_0 \\
      x_0 & x_1 & x_2 & \cdots & x_{m-2} & x_{m-2} & x_{m-2} & x_{m-2} & x_{m-2} & \cdots & x_1 & x_0 \\
      \vdots & \vdots & \vdots & \vdots & \vdots & \vdots & \vdots & \vdots & \vdots & \vdots & \vdots & \vdots \\
      x_0 & x_0 & x_0 & \cdots & x_0 & x_0 & x_0 & x_0 & x_0 & \cdots & x_0 & x_0 \\
    \end{matrix}
  \end{equation*}
  The $n$-th row contains the path $x$ up to $t = m-n$ and then $2n-1$ copies of $x_{m-n}$ follow and after that the paths continuous with $x_{m-n}$ up to $x_0$.
  It is trivial to show that each row is a correct $\theta$-loop. It remains to show that each column is a $\theta$-path. To this end let $i \in [m]$ and define the $i-th$ row as $r: [m] \to X$.
  % TODO: show friedrich martin
  Then if $\forall t \in [m-i]: r(t) = x_t$ and $\forall t \in [m-i+1, m]: r(t) = x_{m-(t-i)}$ and thus $d_X(r(i), r(i+1)) \leq \theta$ and hence $r$ is a $\theta$-path.
\end{proof}

\begin{rem}
  For $\theta' \leq \theta$ there exists a natural homomorphism $\pi_{1,\theta'}(X, x_0) \to \pi_{1,\theta}(X, x_0)$ 
  because every $\theta'$-path is also a $\theta$-path and every $\theta'$-grid homotopy is also a $\theta$-grid homotopy.
\end{rem}

\begin{rem}
  Every closed $\theta$-path with length $\leq 4\theta$ is $\theta$-homotopic to a constant path.
\end{rem}

\begin{proof}
  Consider a closed $\theta$-path $(z_0, z_1, z_2, z_3, z_4 = z_0)$ in $X$. 
  Define the constant path $e_{z_0}: [0] \to X, 0 \mapsto z_0$ and the lazification $Z': [4] \to X, i \mapsto z_0$ of $Z$.

  Then $(z_0, z_1, z_2, z_3, z_4 = z_0) \sim_{\theta} (z_0, z_1, z_1, z_0, z_0)\sim_{\theta} (z_0, z_0, z_0, z_0, z_0) = Z'$. The remark follows from the fact that $\sim_{\theta}$ is an equivalence relation.
\end{proof}

In this sense one can imagine that $\pi_{1,\theta}$ is isomorphic to a factor of $\pi_1$ where loops shorter than $4\theta$ are all in the equivalence class of the constant loop.

\section{Inverse Limit}

The following definition of the inverse limit is the special case in the category of groups. 
But keep in mind that this notion can be generalized to arbitrary categories.
% maybe add inverse system definition
\begin{defin}
  Let $(I, \leq)$ be a directed poset\footnote{Directed partially ordered set}, 
  $(G_i)_{i\in I}$ be a family of groups and let $(\varphi_{ij})_{i,j \in I, i \leq j}$ be a family of group homomorphisms $\varphi_{ij}: G_j \to G_i$ such that $\forall i,j,k \in I$ with $i \leq j \leq k$:
  \begin{equation*}
    \varphi_{ii} = \text{id}_{G_i} \: \land \: \varphi_{ij} \circ \varphi_{jk} = \varphi_{ik}
  \end{equation*}
  Then the \textbf{inverse limit} of the family $(G_i)_{i\in I}$ with respect to the homomorphisms $(\varphi_{ij})_{i,j \in I, i \leq j}$ is defined as:
  \begin{equation*}
    \varprojlim_{i\in I} G_i := \left\{(g_i)_{i\in I} \in \prod_{i\in I} G_i \: \middle| \: \forall i,j \in I, i \leq j: \varphi_{ij}(g_j) = g_i\right\}
  \end{equation*}
  with the group operation $(g_i)_{i\in I}, (h_i)_{i\in I} \mapsto (g_i \cdot h_i)_{i\in I}$.
\end{defin}

The construction of such inverse limits is very useful because it has the following \textit{universal property}:
\begin{figure}[ht!]
  \centering
  % https://tikzcd.yichuanshen.de/#N4Igdg9gJgpgziAXAbVABwnAlgFyxMJZAJgBoAGAXVJADcBDAGwFcYkQAJEAX1PU1z5CKMsWp0mrdgB1pDAE5p5EAFaMsAWwD6wLLKxgABAEluhgOJasPPiAzY8BIuVIBmcQxZtEISypv8DkJEACxuHpLevlY84jBQAObwRKAAZsoaSC4gOBBIAIw0nlI+smjYIDSM9ABGMIwACgKOwiAG2LABIOkQmYjZuUhkEl4y0uVYWv5VtfVNQU4+7VidvGkZWTSDiK5FkWMTMTN1jc3BS2AdbGvdG4iFOXmIw8VRZZPTINUn84KLbZcVtdbD0+g9trsRiUQO8jl9ZqcFq1lqsQXdhhC9qNSnJ6IoABaTXQqbiVeE-M7-FHXSjcIA
  \begin{tikzcd}
    &  & H \arrow[dd, dashed, "\exists!\psi" description] \arrow[llddd, "\psi_j" description] \arrow[rrddd, "\psi_i" description] &  &     \\
    &  &                                                                                                          &  &     \\
    &  & \varprojlim_{i\in I} G_i \arrow[lld, "\pi_j" description] \arrow[rrd, "\pi_i" description]               &  &     \\
  G_j \arrow[rrrr, "\varphi_{ij}" description] &  &                                                                                                          &  & G_i
  \end{tikzcd}
  \caption{Universal property of the inverse limit}\label{fig:inverse-limit}
\end{figure}

\begin{defin}
  Let $(I, \leq)$ be a directed poset, $(G_i)_{i\in I}$ be a family of groups and let $(\varphi_{ij})_{i,j \in I, i \leq j}$ be a family of group homomorphisms $\varphi_{ij}: G_j \to G_i$.
  Addionally let $\pi_i$ be the projection maps from the direct product to the $i$-th component.
  
  Then the inverse limit $\varprojlim_{i\in I} G_i$ has the following universal property:
  If there exits a group $H$ and a family of group homomorphisms $(\psi_i)_{i\in I}: H \to G_i$ such that $\forall i,j \in I, i \leq j: \varphi_{ij} \circ \psi_j = \psi_i$ then there exists a unique group homomorphism $\psi: H \to \varprojlim_{i\in I} G_i$ such that $\forall i \in I: \pi_i \circ \psi = \psi_i$.
  The commutative diagram in figure \ref{fig:inverse-limit} illustrates this property.
\end{defin}

% \pi_{1,\theta} has inverse limit
% universal property says that there is a unique homomorphism from the inverse limit to the discrete fundamental group

In order to use this useful property of the inverse limit we need to define maps from $\pi_1$ to $\pi_{1,\theta}$ for $\forall \theta \in (0,\infty)$. This will be done in the following way:
\begin{defin}
  Let $\theta \in \R_{>0}$, $\gamma: [0,1] \to X$ be a continuous path in $X$ and let $0 = t_0 \leq t_1 \leq \cdots \leq t_n = 1$ for $n \in \N$.

  The sequence $(\gamma(t_i))_{i=0}^n$ is called a \textbf{$\theta$-discretization} of $\gamma$ if $\forall i \in \{1, \cdots, n\}$ it holds that:
  \begin{equation*}
    d_X(\gamma(t_{i}), \gamma(t_{i-1})) \leq \theta
  \end{equation*}
  The discretization of $\gamma$ is denoted as $\widehat{\gamma}^{\theta}_{(t_0, \cdots, t_n)}$.
\end{defin}

This definition is only useful if the following lemma can be proven:

\begin{lemma}\label{lem:discretization}
  Every continuous path admits a unique $\theta$-discretization up to $\theta$-homotopy. If the paths $\alpha, \beta: [0,1] \to X$ are homtopic then any two $\theta$-discretizations 
  $\widehat{\alpha}^{\theta}_{(t_0, \cdots, t_n)}$, $\widehat{\beta}^{\theta}_{(t_0', \cdots, t_m')}$ are $\theta$-homotopic ($n,m \in \N$).
\end{lemma}

% TODO: maybe define B(x, r)
% TODO: Lesbegue number lemma in footnote
\begin{proof}
  Let $\theta \in (0, \infty]$, $\gamma: [0,1] \to X$ be a continuous path in $X$. 
  Let $\mathcal{U} = \left\{\gamma^{-1}\left(B(x,{\theta \over 2})\right) \: \middle| \: x \in X \right\}$ which is an open cover of $[0,1]$.
  By the Lebesgue number lemma there exists a $N \in \N$ such that for every $t$ in $[0,1]$ can be choosen large enough, such that $[t-1/N, 1+1/N] \subseteq U$ for a $U \in \mathcal{U}$.
  Now set $t_i := i/N$ with $i \in \{0, \cdots, N\}$. Then $\forall i \in \{0, \cdots, N\}$ it holds that:
  \begin{align*}
    &\forall i \in \{1, \cdots, N\}: \gamma(t_{i-1}), \gamma(t_i) \in U \\
    \Rightarrow \: &\forall i \in \{1, \cdots, N\}: d_X(\gamma(t_{i-1}), \gamma(t_i)) \leq \sup\limits_{x,y \in U} d_X(x,y) = \theta
  \end{align*}
  which means that $\widehat{\gamma}^{\theta}_{(t_0, \cdots, t_N)}$ is a $\theta$-discretization of $\gamma$.

  Now let $n,m \in \N$ and let $\widehat{\gamma}_{(t_0, \cdots, t_n)}^{\theta}$ and $\widehat{\gamma}_{(t_0', \cdots, t_m')}^{\theta}$ be two $\theta$-discretizations of $\gamma$.
  Wlog it can be assumed that inequality $t_0 < t_1 < \cdots < t_n$ are all strict, $n \leq m$ and that for every $t_i$ there exists a $j \geq i$ such that $t_i = t_j'$.
  Now define the surjective, monotone map $f: [m] \to [n], j \mapsto \max \{t_i \leq t_j'\}$ 
  and thus $(\gamma(t_{f(0)}), \cdots, \gamma(t_{f(m)}))$ is a lazification of $(\gamma(t_0), \cdots, \gamma(t_n))$ and is $\theta$ homotopic to $(\gamma(t_0'), \cdots, \gamma(t_m'))$.
  
  Let $H: [0,1] \times [0,1] \to X$ be a homotopy between $\alpha$ and $\beta$. As above, Lesbegue number lemma can be used to find a $N \in \N$ 
  such that for every pair $(t,s) \in [0,1]^2$, such that $H(B((t,s),1/N)) \subseteq B(x, \theta/2)$.
  Define $\widehat{H}^{\theta}: [N]^2 \to X, (i,j) \mapsto H(i/N, j/N)$. It is easy to see that $\widehat{H}^{\theta}(\cdot, 0)$ is a $\theta$-discretization of $\alpha$ and
  $\widehat{H}^{\theta}(\cdot, N)$ is a $\theta$-discretization of $\beta$ by the argument above and $\widehat{H}^{\theta}(0, \cdot)$ is a (free) $\theta$-discretization betweem them because by definition it holds that:
  \begin{equation*}
    d_X(\widehat{H}^{\theta}(i,j), \widehat{H}^{\theta}(i,j+1)) \leq \theta, d_X(\widehat{H}^{\theta}(i,j), \widehat{H}^{\theta}(i+1,j)) \leq \theta
  \end{equation*}
  for $\forall i, j \in [N-1]$.
\end{proof}

By Lemma \ref{lem:discretization} it follows that the discretization map $\gamma \mapsto \widehat{\gamma}^{\theta}$ is well-defined on the level of paths
and that it induces a map from $\pi_1(X, x_0)$ to $\pi_{1,\theta}(X, x_0)$ because it is easy to see that the map is compatible with the operation of path concatenation.
Furthermore, for $\theta' \leq \theta$ diagram in figure \ref{fig:discretization} commutes.

\begin{figure}[ht!]
  \centering
  \begin{tikzcd}
      &  & {\pi_1(X,x_0)} \arrow[lldd, "\widehat{\cdotp}^{\hspace*{1px}\theta'}" description, shift right] \arrow[rrdd, "\widehat{\cdot}^{\hspace*{1px}\theta}" description] &  &                          \\
      &  &                                                                                                                                        &  &                          \\
    {\pi_{1, \theta'}(X,x_0)} \arrow[rrrr] &  &                                                                                                                                        &  & {\pi_{1,\theta}(X, x_0)}
  \end{tikzcd}
  \caption{Discretization map}\label{fig:discretization}
\end{figure}

Now by the universal property of the inverse limit it follows that there exists a unique group homomorphism $\: \widehat{\cdot}^{\hspace*{2px}\theta}: \pi_{1,\theta}(X, x_0) \to \varprojlim \pi_{1,\theta}(X, x_0)$.

\begin{thm}
  If $X$ is u.l.p.c. and u.s.l.s.c. then the discretization map $\: \widehat{\cdot}^{\hspace*{2px}\theta}: \pi_{1,\theta}(X, x_0) \to \varprojlim \pi_{1,\theta}(X, x_0)$ is an group isomorphism.
\end{thm}

\begin{proof}
\end{proof}

\section{Counterexamples}

\clearpage
\bibliographystyle{alpha}
\bibliography{ref} % see references.bib for bibliography management

\end{document}