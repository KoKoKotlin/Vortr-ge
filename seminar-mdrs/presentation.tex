\documentclass[12pt, aspectratio=169]{beamer}

\usepackage{color}
\usepackage[utf8]{inputenc}
\usepackage{graphicx}
\usepackage{tikz}
\usepackage[T1]{fontenc}
\usepackage{caption}
\usepackage{wrapfig}
\usepackage{xurl}
\usepackage{animate}

\usepackage{amsmath}
\usepackage{amssymb}
\usepackage{amsthm}
\usepackage{babel}[ngerman]
\usepackage{hyperref}
\usepackage{colonequals}
\usepackage{xfrac}
\usepackage{tikz}
\usepackage{tikz-cd}

\usepackage{graphicx}
\usepackage{caption}

\captionsetup[figure]{labelformat=empty}
\setbeamertemplate{section in toc}[square]
\usefonttheme[onlymath]{serif}

\usetheme{Singapore}
\usecolortheme{default}
\usefonttheme{structurebold}
\setbeamerfont{text}{size=\large}
\setbeamertemplate{bibliography item}{\insertbiblabel}

\definecolor{ao}{rgb}{0.0, 0.0, 0.7}


\setbeamercolor{block body alerted}{bg=alerted text.fg!10}
\setbeamercolor{block title alerted}{bg=alerted text.fg!20}
\setbeamercolor{block body}{bg=structure!10}
\setbeamercolor{block title}{bg=structure!20}
\setbeamercolor{block body example}{bg=green!10}
\setbeamercolor{block title example}{bg=green!20}

\newcommand{\nimadd}{\overset{*}{+}}

%\setbeamertemplate{footline}[frame number]
\setbeamertemplate{footline}[text line]{%
  \parbox{\linewidth}{\vspace*{-8pt}\color{ao}\insertshorttitle\hspace{10px}\insertshortauthor\hfill\insertpagenumber}}

\newcommand{\R}[0]{\mathbb{R}}
\newcommand{\N}[0]{\mathbb{N}}
\newcommand{\T}[0]{\mathcal{T}}
\newcommand{\NB}[0]{\mathcal{N}}
\newcommand*{\logeq}{\ratio\Leftrightarrow}
\newcommand*{\longeq}{\ratio\Longleftrightarrow}
\newcommand*{\bigdot}{\mathpalette\bigcdot@{.5}}
\DeclareRobustCommand{\loongrightarrow}{%
\DOTSB\relbar\joinrel\relbar\joinrel\rightarrow
}

\title{Seminarvortrag \\Fundamentalgruppen als Limites diskreter Fundamentalgruppen}
\author[Y. Höll]{Yannik Höll}
\date{12. Juni, 2024}
% \logo{\includegraphics[keepaspectratio=True, width=30px]{./image/logo.png}}

\beamertemplatenavigationsymbolsempty 

\begin{document}
\begin{frame}[noframenumbering, plain]
	\titlepage
\end{frame}

\begin{frame}
	\frametitle{Einteilung}
	\tableofcontents
\end{frame}

\begin{frame}
    Notation:
        \begin{itemize}
            \item $Y$ ist ein beliebiger topologischer Raum mit Tologie $\mathcal{T}_Y$,
            \item $X$ ist ein beliebiger metrischer Raum mit Metrik $d_X$,
            \item Für $Z_1, Z_2$ topologische Räume ist $C(Z_1, Z_2) := \{f: Z_1 \to Z_2 \: | \: f \text{ stetig}\}$
            \item Für $y \in Y$ ist $\NB_y$ der Nachbarschaftsfilter von $y$ in $Y$,
            \item Für $\varepsilon \in \R_{>0}$ und $x \in X$ ist $B_{\varepsilon}(x)$ die offene Kugel um $x$ mit Radius $\varepsilon$,
            \item Für $n \in \N$, sei $[n] := \{0, 1, \ldots, n\}$,
            \item Für $n,m \in \N$ mit $n \leq m$, sei $[n,m] := \{n, n+1, \ldots, m\}$.
        \end{itemize}        
\end{frame}

\section{Topologische Grundlagen}

\begin{frame}
    \begin{definition}
        Ein \textbf{Weg} in $Y$ ist eine stetige Abbildung von $[0,1]$ nach $Y$ wobei $[0,1]$ die euklidische Metrik von $\R$ übernimmt.
    \end{definition}
    \begin{definition}<2->
        \onslide<2-> {
        Let $\gamma_1, \gamma_2 \in C([0, 1], Y)$ mit selbem Start- und Endpunkt. Eine \textbf{Homotopie} $H\colon [0,1]^2 \to Y$ zwischen $\gamma_1$ und $\gamma_2$ ist eine stetige Abbildung für die gilt
        \begin{align*}
            &\forall s \in [0,1]\colon H(s, 0) = \gamma_1(s) \: \land \: H(s, 1) = \gamma_2(s), \\
            &\forall t\in[0,1]\forall i\in \{1,2\}\colon H(0, t) = \gamma_i(0) \: \land \: H(1, t) = \gamma_i(1).
        \end{align*}
        }
        \onslide<3-> { Die Wege $\gamma_1, \gamma_2$ heißen \textbf{homotop} ($\gamma_1 \sim_H \gamma_2$), falls eine Homotopie $H$ zwischen ihnen existiert. }
        \onslide<4-> { Wenn ein Weg homotop zum konstanten Weg ($[0,1] \to Y, t \mapsto y \in Y$) ist, dann heißt er \textbf{null-homotop}.}
    \end{definition}
\end{frame}

\begin{frame}
    \begin{definition}<1->
        Die \textbf{Homotopierelation} $\sim$ auf $C([0,1], Y)$ ist wie folgt definiert 
        \begin{equation*}
          \forall\gamma_1, \gamma_2 \in C([0,1], Y)\colon \gamma_1 \sim \gamma_2 \iff \exists H \text{ Homotopie zwischen } \gamma_1 \text{ und } \gamma_2.
        \end{equation*}
    \end{definition}
    \begin{lemma}<2->
        Die Homotopierelation $\sim$ von Wegen ist eine Äquivalenzrelation auf der Menge $C([0,1],Y)$.
    \end{lemma}
\end{frame}

% TODO: add source
\begin{frame}
    \centering
    \animategraphics[loop,controls,width=.5\textwidth]{10}{homotopie/hom-}{0}{50}
\end{frame}

\begin{frame}
\begin{definition}
    \begin{enumerate}
        \item $Y$ \textbf{zusammenhängend} $\;:\Longleftrightarrow\; \forall A,B \in \T_Y\colon (A \cap B = \emptyset \: \land \: Y = A \cup B) \Rightarrow (A = \emptyset \: \lor \: B = \emptyset)$,
        \item $Y$ \textbf{weg-zusammenhängend} $\;:\Longleftrightarrow\; \forall x,y \in Y \exists\gamma \in C([0,1], Y)\colon \gamma(0) = x \: \land \: \gamma(1) = y$,
        \item $Y$ \textbf{lokal weg-zusammenhängend} $\;:\Longleftrightarrow$ \\ $\forall y \in Y\forall U \in \NB_y\exists V\in \NB_y\colon V \subseteq U \: \land \: V$ (weg-)zusammenhängend,
        \item $Y$ \textbf{semi-lokal einfach zusammenhängend} (s.l.s.c) $\: :\Longleftrightarrow$ \\ $\forall y\in Y\exists U \in \NB_y\colon$ Jede Schleife in $U$ ist null-homotop in $Y$
    \end{enumerate}
\end{definition}
\end{frame}

\begin{frame}
Uniforme Bediengungen für metrische Räume:    
\begin{definition}
    \begin{enumerate}
        \item $X$ \textbf{uniform lokal weg-zusammenhängend} (u.l.p.c.) $\: :\Longleftrightarrow$ \\ $\forall \varepsilon \in \R_{>0} \exists \delta \in \R_{>0}\forall x \in X\colon \forall x_1, x_2 \in B_{\delta}(x)\exists \gamma \in C([0,1], B_{\varepsilon}(x))\colon$ \\ $\gamma(0) = x_1 \: \land \: \gamma(1) = x_2$,
        \item $X$ \textbf{uniform semi-lokal einfach zusammenhängend} (u.s.l.s.c) $\: :\Longleftrightarrow$ \\ $\exists \varepsilon \in \R_{>0}\forall x\in X\colon$ Jede Schleife in $B_{\varepsilon}(x)$ ist null-homotop in $X$
        \end{enumerate}
\end{definition}
\end{frame}

\begin{frame}
    \begin{definition}
        \onslide<1-> {
            Sei $y_0 \in Y$. Die \textbf{Fundamentalgruppe} von $Y$ mit Basispunkt $y_0$ ist definiert als
            \begin{equation*}
            \pi_1(Y, y_0) := (\{\gamma \in C([0,1],Y) \: | \: \gamma(0) = \gamma(1) = y_0\}/_{\sim}, \: *)
            \end{equation*}
        }
        \onslide<2-> {
        wobei 
        \begin{align*} 
          &*\colon C([0,1], Y) \times C([0,1], Y) \to C([0,1], Y), \\
          &(\gamma_1, \gamma_2) \mapsto \left(t \mapsto \begin{cases}
          \gamma_1(2t),   &t \in [0, {1 \over 2}], \\
          \gamma_2(2t-1), &t \in [{1 \over 2}, 1],
            \end{cases}\right)
        \end{align*}
        die \textbf{Konkatenation} von Pfaden ist. 
        Da nur Schleifen mit dem selben Basispunkt betrachtet werden, kann man belibiege Elemente von $\pi_1(Y, y_0)$ konkatenieren.
        }    
    \end{definition}
\end{frame}

\begin{frame} \frametitle{Lemma von Lebesgue}
    \begin{lemma}
        Sei $\mathcal{U}$ eine offene Überdeckung von $X$. Wenn $X$ kompakt ist, dann existiert ein $\delta \in \R_{>0}$, sodass
        \begin{equation*}
          \forall A \in \mathcal{P}(X)\colon d(A) < \delta \Rightarrow (\exists U \in \mathcal{U}\colon A \subseteq U),
        \end{equation*}
        wobei $d\colon \mathcal{P}(X) \to [0, \infty), \: A \mapsto \sup \{ d_X(a_1, a_2) \: | \: a_1,a_2\in A \}$ der Durchmesser einer Teilmenge von $X$ ist.
        Diese Zahl $\delta$ wird dann als \textbf{Lebesguezahl} bezeichnet.
    \end{lemma}
\end{frame}

\section{Quellen}
\begin{frame}[allowframebreaks, noframenumbering]
    \nocite{*}
	\hfill
    \bibliographystyle{unsrt}
    \bibliography{presentation-ref}
\end{frame}


\end{document}