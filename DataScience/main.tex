\documentclass{article}
\usepackage[left=3cm,right=3cm,top=2cm,bottom=2cm]{geometry}
\usepackage[ngerman]{babel}
\usepackage{amsmath}
\usepackage{amssymb}
\usepackage{amsthm}

\setlength{\parindent}{0mm}

\newcommand{\R}[0]{\mathbb{R}}
\newcommand{\F}[0]{\mathcal{F}}

\title{Datascience}
\author{Yannik Höll}
\date{\today}

\newtheorem{defin}{Definition}
\newtheorem{thm}{Satz}

\begin{document}
\maketitle

\section{Inverse Faltung}

\subsection{Klassische Fouriertransformation und Faltung}

Für folgende Definitionen sei $n \in \mathbb{N} \setminus \{ 0 \}$.

\begin{defin}
    Seien $f,g \in L^1(\R^n)$. \\
    Dann ist die \textbf{Konvolution} von $f$ mit $g$ für $x \in \Omega$ definiert als:
    \begin{equation}
        (f \star g)(x) := \int_{\R^n} f(x - y) g(y) dy
    \end{equation}
\end{defin}

\begin{defin} Die \textbf{Fourier-Transformation} einer Funktion $f \in L^1(\R^n)$ ist definiert durch:
    \begin{equation}
        (\mathcal{F}f)(\xi) := \int_{\R^n} f(x) \cdot e^{-2\pi i x \cdot \xi} dx
    \end{equation}
    für $x, \xi \in \R^n$. Dabei ist $x \cdot \xi$ das Skalarprodukt im $\R^n$.
\end{defin}

Die Fouriertransformation ist wohldefiniert, weil $f \in L^1(\R^n)$.
\begin{proof}
    Sei $f \in L^1(\R^n)$, dann gilt:
    \begin{align*}
        \Bigg\lvert \int_{\R^n} f(x) \cdot e^{-2\pi i x \cdot \xi} dx \Bigg\rvert &\leq \int_{\R^n} \lvert f(x) \rvert \lvert e^{-2\pi i x \cdot \xi} \rvert dx \\
        &= \int_{\R^n} \lvert f(x) \rvert dx \\
        &= \lVert f \rVert_{L^1(\R^n)} < \infty
    \end{align*}
\end{proof}

\begin{defin}
    Die \textbf{Inverse Fouriertransformation} einer Funktion $f \in L^1(\R^n)$ ist definiert durch:
    \begin{equation}
        (\mathcal{F}f)(\xi) := \int_{\R^n} f(x) \cdot e^{2\pi i x \cdot \xi} dx
    \end{equation}
    für $x, \xi \in \R^n$. Dabei ist $x \cdot \xi$ das Skalarprodukt im $\R^n$.
\end{defin}
Der Nachweis der Wohldefiniertheit funktioniert genauso, wie bei der Fouriertransformation.

\begin{thm}
    \textbf{Faltungssatz} \\
    Seien $f,g \in L^1(\R^n)$ und $\mathcal{F}f \cdot \mathcal{F}g \in L^1(\R^n)$. Dann gilt:
    \begin{equation}
        \mathcal{F}(f \star g) = \mathcal{F}f \cdot \mathcal{F}g
    \end{equation}

    \begin{proof}
        Der Beweis ist abgeschlossen, wenn gezeigt wurde, dass $\F^{-1} (\F f \cdot \F g) = f \star g$. 
        
        Sei $F(\omega) := \F f(\omega)$ und $G(\omega) := \F g(\omega)$. Dann gilt: \\
        \begin{align*}
            \F^{-1} (F(\omega) G(\omega)) &= \int_{\R^n} F(\omega) G(\omega) \cdot e^{2\pi i \omega \cdot t} d\omega \\
            &= \int_{\R^n} \Bigg\lbrack \int_{\R^n} f(s) \cdot e^{-2\pi i s \cdot \omega} ds \Bigg\rbrack G(\omega) \cdot e^{2\pi i \omega \cdot} d\omega \\
            &= \int_{\R^n} \int_{\R^n} f(s) G(\omega) \cdot e^{2\pi i \omega \cdot (s - t)} ds d\omega
        \end{align*}
        Nach dem Satz von Fubini kann die Integrationsreihenfolge vertauscht werden:
        \begin{align*}
            \F^{-1} (F(\omega) G(\omega)) &= \int_{\R^n} \int_{\R^n} f(s) G(\omega) \cdot e^{2\pi i \omega \cdot (s - t)} d\omega ds \\
            &= \int_{\R^n} f(s) \Bigg \lbrack \int_{\R^n} G(\omega) \cdot e^{2\pi i \omega \cdot (t - s)} d\omega \Bigg \rbrack ds \\
            &= \int_{\R^n} f(s) g(t-s) ds = f \star g
        \end{align*}
    \end{proof}
\end{thm}

\subsection{Diskrete Fouriertransformation}

Mit Satz 1 ist es nun theoretisch möglich, die Faltung von zwei Funktionen umzukehren, unter der Vorraussetzung, 
dass eine der beiden Funktionen bekannt ist. Dafür kann man die Fouriertransformation wie folgt nutzen:

Seien $w: \R \to \R$ und $f, g \in L^1(\R)$, sodass gilt: $w = f \star g$. Dabei sind $w$ und $g$ bekannt und $f$ ist die gesuchte Funktion.
Wenn $w$ und $g$ hinreichund gute Eigenschaften haben, sodass die Integrale existieren und die Inverse Fouriertransformation existiert, dann kann
man folgenden Algorithmus nutzen, um $f$ zu berechnen:
\begin{enumerate}
    \item Berechne $\hat{w} := \mathcal{F}w$ und $\hat{g} := \mathcal{F}g$.
    \item Berechne $\hat{f} := \frac{\hat{w}}{\hat{g}}$.
    \item Berechne $f := \mathcal{F}^{-1}\hat{f}$.
\end{enumerate}

Durch die Berechnung im 2. Schritt wird sofort deutlich, dass $\hat{g}$ keine Nullstellen besitzen darf,
da ansonsten der Quotient nicht existiert. Zusätzlich muss man auch über Stetigkeitseigenschaften der Methode nachdenken,
da man Komponenten von hohen Frequenzen von $w$ potentiell stark aufgebläht werden 
und damit auch potentielle Messfehler verstärt werden können.

Für konkrete Berechnungen benötigt man die diskrete Fouriertransformation. 
Sei dafür $u \in C^0(\R)$ mit $supp(u) := \{x \in \R \: | \: u(x) \neq 0 \} \subseteq [-a, a], a > 0$. Diese Funktion wird an $N \in \mathbb{N}_{>0}$ äquidistanten Stellen betrachtet, 
die z.B. Messwerte darstellen können. Sei dafür $u_j = u(t_j), t_j := jh, h:= \frac{2a}{N}, j \in \{-\frac{N}{2}, \cdots, \frac{N}{2} - 1\}$.

Dann kann $u$ durch lineare B-Splines approximiert werden durch:
\begin{equation}
    B_2(t) := \begin{cases}
        t + 1, &-1 \leq t \leq 0, \\
        1 - t, &0 \leq t \leq 1, \\
        0, &{\rm sonst}.
    \end{cases} \\
\end{equation}
\begin{equation}
    u_N(t) := \sum\limits_{j=-N/2}^{N/2-1} u_j B_2 \left(\frac{t}{h} - j\right)
\end{equation}

Durch Fouriertransformation erhält man:
\begin{equation}
    \mathcal{F}(u_N)(y) := \left(\frac{{\rm sin}(\pi h y)}{\pi h y}\right) \cdot \underbrace{\left(h \sum\limits_{j=-N/2}^{N/2-1} u_j e^{-2\pi i jhy}\right)}_{:=U(y)}
\end{equation}

Wählt man $y := \frac{k}{2a}, \: k \in \mathbb{Z}$, dann sind die Stellen, an denen man die Funktion $u$ und die Transformierte $\mathcal{F}(u_N)$
auswertet äquidistant. Damit erhält man schlussendlich:

\begin{align}
    U_k = \frac{1}{N} \sum\limits_{j=-N/2}^{N/2-1} u_j e^{-2\pi ijk/N}
\iff u_j = \frac{1}{N} \sum\limits_{k=-N/2}^{N/2-1} U_k e^{2k\pi ijk/N}
\end{align}

Dabei sind die $U_k$ nun die diskreten Fourierkoeffizienten von $u$ zu den entsprechenden Stützstellen.

Diese diskrete Variante der Transformation kann nun verwendet werden, um die Inverse Faltung wie oben beschrieben für konkrete Messwerte zu berechnen. 
Es handelt sich allerdings um ein schlecht gestelltes Problem und man muss Regularisierungen anwenden, um die die schlechten Stetigkeitseigenschaften der Methode zu verbessern. 

\section{PLSR}

\subsection{Singulärwert-Zerlegung (SVD)}
\subsection{Principle Component Analysis (PCA)}

\end{document}