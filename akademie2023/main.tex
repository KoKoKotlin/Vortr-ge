\documentclass[12pt, aspectratio=169]{beamer}

\usepackage{color}
\usepackage[utf8]{inputenc}
\usepackage{graphicx}
\usepackage{tikz}
\usepackage[T1]{fontenc}
\usepackage{caption}
\usepackage{wrapfig}
\usepackage{xurl}

\captionsetup[figure]{labelformat=empty}
\setbeamertemplate{section in toc}[square]

\usetheme{Singapore}
\usecolortheme{default}
\usefonttheme{structurebold}
\setbeamerfont{text}{size=\large}
\setbeamertemplate{bibliography item}{\insertbiblabel}

\definecolor{ao}{rgb}{0.0, 0.0, 0.7}

%\setbeamertemplate{footline}[frame number]
\setbeamertemplate{footline}[text line]{%
  \parbox{\linewidth}{\vspace*{-8pt}\color{ao}\insertshorttitle\hspace{10px}\insertshortauthor\hfill\insertpagenumber}}

\title{Kombinatorische Spieltheorie}
\author[Y. Höll]{Yannik Höll}
\date{3. März, 2023}
% \logo{\includegraphics[keepaspectratio=True, width=30px]{./image/logo.png}}

\beamertemplatenavigationsymbolsempty 

\begin{document}
\begin{frame}[noframenumbering, plain]
	\titlepage
\end{frame}

\begin{frame}
	\frametitle{Einteilung}
	\tableofcontents
\end{frame}

\section{Einleitung}
\begin{frame}
Kombinatorische Spieltheorie beschäftigt sich mit Spielen die:

\begin{itemize}
    \item 2 Spieler (Links, Rechts)
    \item endliche/abzählbare Positionen
    \item Spieler ziehen abwechselnd
    \item jeder Spieler hat vollständige Information
    \item kein Zufall
    \item Konvention: Keine Züge mehr $\Rightarrow$ Verlierer (kein Unentschieden)
\end{itemize}

\onslide<2-> {
Kombinatorische Spieltheorie ist:
\begin{itemize}
    \item nicht wie gewöhnliche Spieltheorie
    \item eher mathematische Rätsel, Denkaufgaben
\end{itemize}
}
\end{frame}

\begin{frame}
Nicht untersucht werden können Spiele wie:
\begin{itemize}
    \item<2-> Schach (Unentschieden)
    \item<3-> Backgammon (Zufall)
    \item<4-> Tennis, oder andere Sportarten (keine diskreten Zustände)
    \item<5-> Schere-Stein-Papier (keine vollständige Information, nicht abwechselnd)
\end{itemize}

\onslide<6-> {
    $\Rightarrow$ viele untersuchte Spiele eher unbekannt \\
    $\Rightarrow$ Spiele wurden wegen Theorie "erfunden"
}
\end{frame}

\section{Nim}
\section{Poker-Nim}
\section{Hackenbush}
\section{Hackenbush-Hotchpotch}

\end{document}